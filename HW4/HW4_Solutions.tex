\documentclass[12pt]{article}
\usepackage[margin=1in]{geometry}  % Sets all margins to 1 inch
\usepackage{amsmath}
\usepackage{amssymb}
\usepackage{amsthm}
\usepackage{xcolor}
\usepackage{listings}
\usepackage{upgreek}
\usepackage{enumitem}
\setlength{\parindent}{0pt}  % Disable paragraph indentation globally

% Define Python style
\lstdefinelanguage{Python}{
    sensitive=true,
    morekeywords={def,return,if,while,None,raise},
    morestring=[b]",
    morestring=[b]',
    morecomment=[l]\#,
}

\lstdefinestyle{python}{
    language=Python,
    basicstyle=\ttfamily\small,
    numbers=left,
    numberstyle=\tiny,
    stepnumber=1,
    numbersep=5pt,
    backgroundcolor=\color{white},
    showspaces=false,
    showstringspaces=false,
    showtabs=false,
    frame=single,
    tabsize=4,
    captionpos=b,
    breaklines=true,
    breakatwhitespace=true,
    breakautoindent=true,
    keywordstyle=\color{blue},
    commentstyle=\color{green!60!black},
    stringstyle=\color{orange}
}

\title{Homework 4}
\author{Eli Bullock-Papa}
\date{\today}

\begin{document}
\maketitle

\section*{Problem 1}
Determine the involutory keys in the affine cipher mod 26.

\vspace{1em}
\textbf{Solution:}

For the affine cipher with key $(a,b)$, the encryption function is:
\[E(x) = ax + b \bmod 26\]

But for involutory keys, the encryption function is the same as the decryption function, so encrypting twice returns the original message, meaning:
\[E(E(x)) = x\]

So we do this for the encryption function:
\[E(E(x)) = a(ax + b) + b \bmod 26 = a^2x + ab + b \bmod 26 \equiv x \pmod{26}\]

Then subtracting $x$ from both sides, we get:
\[a^2x + ab + b - x \equiv 0 \pmod{26}\]

Then grouping the terms, we get:
\[(a^2 - 1)x + (ab + b) \equiv 0 \pmod{26}\]

Now we know that both:
\[a^2 - 1 \equiv 0 \pmod{26}\]
\[ab + b \equiv 0 \pmod{26}\]

First, let's solve for $a$ in the equation $a^2 - 1 \equiv 0 \pmod{26}$ we must check for all possible values of $a$ (note that $a$ must be coprime to 26):
\begin{align*}
1^2 &\equiv 1 \pmod{26} \quad \checkmark\\
3^2 = 9 &\not\equiv 1 \pmod{26}\\
5^2 = 25 &\not\equiv 1 \pmod{26}\\
7^2 = 49 \equiv 23 &\not\equiv 1 \pmod{26}\\
9^2 = 81 \equiv 3 &\not\equiv 1 \pmod{26}\\
11^2 = 121 \equiv 17 &\not\equiv 1 \pmod{26}\\
15^2 = 225 \equiv 17 &\not\equiv 1 \pmod{26}\\
17^2 = 289 \equiv 3 &\not\equiv 1 \pmod{26}\\
19^2 = 361 \equiv 23 &\not\equiv 1 \pmod{26}\\
21^2 = 441 \equiv 25 &\not\equiv 1 \pmod{26}\\
23^2 = 529 \equiv 9 &\not\equiv 1 \pmod{26}\\
25^2 = 625 \equiv 1 &\equiv 1 \pmod{26} \quad \checkmark
\end{align*}

If we assume that $a = 1$ in the equation $ab + b \equiv 0 \pmod{26}$:
\[(1)b + b \equiv 0 \pmod{26}\]
\[2b \equiv 0 \pmod{26}\]
\[b \equiv 0 \pmod{26}\ or \ b \equiv 13 \pmod{26}\]

If we assume that $a = 25$ in the equation $ab + b \equiv 0 \pmod{26}$:
\[(25)b + b \equiv 0 \pmod{26}\]
\[26b \equiv 0 \pmod{26}\]
\[b \in \mathbb{Z}_{26}\]

Therefore, the complete set of involutory keys $(a,b)$ for the affine cipher mod 26 is:
\[\{(1,0), (1,13)\} \cup \{(25,b) : b \in \mathbb{Z}_{26}\}\]

\newpage
\section*{Problem 2}
Let $\sigma = \begin{pmatrix} 
1 & 2 & 3 & 4 & 5 & 6 & 7 & 8 & 9 \\ 
7 & 3 & 4 & 2 & 8 & 6 & 5 & 9 & 1 
\end{pmatrix}$

\textbf{(a) Write $\sigma$ in one-row (cycle) notation.}
\[\sigma = (1 \; 7 \; 5 \; 8 \; 9)(2 \; 3 \; 4)(6)\]

\vspace{1.5em}
\textbf{(b) What is the cycle type of $\sigma$?}

The cycle type of $\sigma$ is $[5,3,1]$.

\vspace{1.5em}
\textbf{(c) Write $\sigma$ as a composition of (not necessarily disjoint) transpositions.}

\[\sigma = (1 \; 9)(1 \; 8)(1 \; 5)(1 \; 7)(2 \; 4)(2 \; 3)\]

\vspace{1.5em}
\textbf{(d) Is $\sigma$ an even permutation or an odd permutation?}

Even, as it is composed of 6 transpositions.

\vspace{1.5em}
\textbf{(e) Is $\sigma$ an involution?}

No, as it violates the condition that the cycle decomposition must consist of only transpositions and fixed points.

\vspace{1.5em}
\textbf{(f) Now, let $\alpha = (1\,4\,8\,5)(1\,3\,6\,7)(2\,9)$. Compute the conjugate of $\sigma$ by $\alpha$.}

Since these are not disjoint cycles, we can simplify into one mapping:


\begin{center}
Let $\alpha = \begin{pmatrix} 
    1 & 2 & 3 & 4 & 5 & 6 & 7 & 8 & 9 \\ 
    3 & 9 & 6 & 8 & 1 & 7 & 4 & 5 & 2 
\end{pmatrix}$
\end{center}

That can also be written as:
\[\alpha = (1 \; 3 \; 6 \; 7 \; 4 \; 8 \; 5)(2 \; 9)\]

Now we assemble $\alpha\sigma\alpha^{-1}$:
\[\sigma = (1 \; 7 \; 5 \; 8 \; 9)(2 \; 3 \; 4)(6)\]
\[\alpha^{-1} = (9 \; 2)(5 \; 8 \; 4 \; 7 \; 6 \; 3 \; 1)\]
\[\alpha\sigma\alpha^{-1} = (1 \; 3 \; 6 \; 7 \; 4 \; 8 \; 5)(2 \; 9)(1 \; 7 \; 5 \; 8 \; 9)(2 \; 3 \; 4)(6)(9 \; 2)(5 \; 8 \; 4 \; 7 \; 6 \; 3 \; 1)\]

We can use the trick from Theorem 5.1 to match the cycles in $\sigma$ to the conjugate and get:
\[\alpha\sigma\alpha^{-1} = (3 \; 4 \; 1 \; 5 \; 2)(9 \; 6 \; 8)(7)\]

\vspace{1.5em}
\textbf{(g) Is $\sigma\alpha\sigma\alpha^{-1}$ of matched cycle type? Explain why or why not.}

\[\sigma = (1 \; 7 \; 5 \; 8 \; 9)(2 \; 3 \; 4)(6)\]
\[\alpha\sigma\alpha^{-1} = (3 \; 4 \; 1 \; 5 \; 2)(9 \; 6 \; 8)(7)\]
\[\sigma\alpha\sigma\alpha^{-1} = (1 \; 7 \; 5 \; 8 \; 9)(2 \; 3 \; 4)(6)(3 \; 4 \; 1 \; 5 \; 2)(9 \; 6 \; 8)(7)\]
\[\sigma\alpha\sigma\alpha^{-1} = (1 \; 8)(2 \; 4 \; 7 \; 5 \; 3)(6 \; 9)\]

The cycle type is $[5, 2, 2]$ which is not matched (no match for the cycle of length 5).

\end{document}
